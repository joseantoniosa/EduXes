\chapter{Description of the practicum}
%\thispagestyle {empty}
The main objective of this Master Thesis consist in the development of an mobile application to be used int highschools by teachers.
It could allows teachers to carry on control students attendance, their behavior. Also it permits quick assessment by activity.
Teachers would read students reports: weekly and daily assessment, by activity assessments and total marks.
\begin{itemize}
  \item {Name :} Jos\'e Antonio Salgueiro Aquino.
  \item {Birth date: } august 5th 1970
\item {Education:} B.Sc. in Fundamental Physics, University of Santiago de Compostela University 1988-1993.
\item {Address:} Marin (Pontevedra). Spain.
\item {Current position: } Secondary School teacher in Technology.
\end{itemize}

\begin{itemize}
  \item {Working times :} (planned) 300 hours. 	From 6th August, to 30 September, on an eight hours day basis.
\end{itemize}
Working times (planned): 300 hours.

This application involves several technologies:
Java language.
Android skeleton application.
PhoneGap framework to develop multi-platform applications.
JQuery and JqueryMobile  to develop mobile oriented applications.
JavaScript with Web Databases.
Git for version control system.
Meetings:
	One meeting on August: Technologies to be used were stated, work methodologies, first application windows (pages), Android version to be used (2.3.3) because is the most popular. 
	Several emails and gtalk conversations about organization, general problems were written.
Teleworking is done. 
Materials and special equipment used:
	For main development: 
Hardware: Intel Quad, 6GiB RAM, 500GiB HD. 

Software: Debian Linux Wheezy (testing), Eclipse Juno, JQuery 1.8.1, jQueryMobile 1.1.1, and PhoneGap-Cordova 1.8.1, Android Virtual Manager 2.3.3, Git 1.7.10.4-1.
For testing Sony-Ericsson Xperia V mobile phone, with USB cable.
    
    
