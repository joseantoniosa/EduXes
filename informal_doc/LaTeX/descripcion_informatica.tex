\chapter{Description}


Preparation of development:
 a) Build development environment: 
Java Development Kit (JDK) version 1.6 is downloaded from: 
http://www.oracle.com/technetwork/java/javase/downloads/index.html
As root user that file is unpacked into /usr/lib/jvm and configured to be the Java default:
\begin{verbatim}
 # update-java-alternatives -s JDK_1.6_NAME
\end{verbatim}  
	Eclipse Juno (4.6) is downloaded from its web-page:
http://www.eclipse.org Download->Linux 64 bits
Android Development Toolkit (ADT) is downloaded following instructions on this page:
http://developer.android.com/sdk/installing/installing-adt.html 
	A new line is included into repository software (Help ->  Install New Software -> Add):
http://dl-ssl.google.com/android/eclipse/
	Next step is to select all the related software listed.
	For Aptana Plugin line to add into Eclipse is: 
http://download.aptana.com/studio3/plugin/install
	Furthermore JQuery, JQueryMobile and Phonegap are needed, and were downloaded from their web sites:
JQuery 1.8.1 (no newer versions):
http://jquery.com/ 
JQuery will be  copied into assets/www/js folder.
JQueryMobile version 1.1.1 from
http://jquerymobile.com/ 
JQueryMobile is a zip file which will be uncompressed and copied into 
assets/www/js folder.
	PhoneGap - Cordova 1.8.1 is downloaded from this URL:
https://github.com/phonegap/phonegap/zipball/1.8.1 
	To create a PhoneGap application this very important instructions (Getting Started with Android)  should be followed step by step:
\cite{AndroidGettingStarted}

 b) Choose application name and folder's policy:
EduXes stands for "Educaci\'n" and "Xesti\'on", is a educational management software.
A folder is created (assets/www/js) which contains javascript (*.js) files except JQuery and JqueryMobile which is included into another
 folder (assets/www/js/jquery), do not forget style sheet files (*.css)
 c) Make a simple application: only a blank page.
Getting started with Android is followed step by step.
 d) Upload simple application into a git repository. 
A Github account is created and a new application is initialized.  This are the source code project page:\cite{EduXes}
Source code are upload to GitHub: 


  \begin{bclogo}[couleur=green!30,arrondi=0.1, logo=\bcpanchant,  ombre=true ] 
{Git init shell}   
\begin{verbatim}
$ git init
$ git add -A *
$ git remote add  EduXes git@github.com:joseantoniosa/EduXes.git 
$ git push origin master
\end{verbatim}
\end{bclogo}




Every time an update is going to be uploaded:
% \begin{verbatim}
% $ git add -A *
% $ git commit -m 'CHANGES_DESCRIPTION'
% $ git push origin master
% 
%  e)  JQueryMobile1 applications are structured in pages (<div data-role=\u201dpage\u201d>), which are very similar to desktop applications windows, therefore, from Javascript code, to change to a new page    $.mobile.changePage("#daily_work")  opens daily_work p page.  
% Application Work-flow 
% Next illustration try to be self-explicative.  Beginning at onDeviceReady()  from interface.fs file.  Inside each page several actions are performed: e.g., open and populate database, load list of groups (loadSchedule()),  and user choose next step according options shown. 
% 
% 
% %%%%%%%%%%%%%%%%%%%%%%%%%%%%%%%%%
% 
% 
% Development:
% There are two JavaScript files:
% - interface.js: It contains information and decisions related to interface and application workflow, completely independent from database.
% - database.js: It contains database related code: SELECT, INSERT, etc. 
% There is only one HTML file:
% index.html only contains HTML framework, page properties, and static content.  
% There are three important files which contain documentation:
% - TODO.txt. List of goals to be achieved and milestone reached.
% - DATABASE.sql  Data-base structure in SQL format.
% - REAME.txt. Only contains general information about this application.
% Next step in development is populate database.
%  f) Populate database with sample data. To test application, sample data are needed. 
%  g) Groups: These pages are not active.
% Make a list of groups window (aka page).
% Make a group management window.
%  h) Students: These pages are not active.
% Make list of students window.
% Students management window (insert-update-delete students)
%  i) Timetable for actual date: list of groups for selected day.
% Below queryScheduleSuccess() and querySchedulePerDayDB() functions are written, these functions fills daily_schedule page as shown in Illustration 6: Application Skeleton.
% 
% %%%%%%%%%%%%%%%%%%%%%%%%%%%%%%%%%%%
% 
% 
% 
% 
% 
% 
% /*
%  *  Main Window
%  */
% function queryScheduleSuccess(tx, results) {
%  var len = results.rows.length;
%    $('#groups_day_ul').empty();
%    var html;
%    var id=0;
%    var description="";
%    var start = "";
%    var t_id_session=-1;
%    for (var i=0;i<len;i++) {
%      id = results.rows.item(i).id;
%      t_id_session = results.rows.item(i).t_id_session;
%      start = results.rows.item(i).s_h_start;  
%      description = results.rows.item(i).description; 
%      html = "<li>";
%      html += "<div data-role='fieldcontain'>";
%      html += start;
%   html += "<a data-role='button' data-iconpos='notext' style='float: right;' href='index.html#list_students_attendance' ";
%    html += " onClick=\"listStudentsAttendance(" + results.rows.item(i).t_id_group + ","+t_id_session + ");\">" + description + "</a>";
%     html += "";
%     html += "</div>";
%     html += "</li>";
%    $('#groups_day_ul').append(html);
%    }
%    $('#groups_day_ul').listview('refresh');
%    }
% 
% /* Query groups per day - Main Window -
% */
% function querySchedulePerDayDB(tx){
%  var query = "SELECT teacher_schedule.id_session, teacher_schedule.day, teacher_schedule.id_group as t_id_group, teacher_schedule.id_session as t_id_session, ";
%   query += " groups.id as g_id, groups.data as description, sessions.id as s_id, sessions.h_start as s_h_start, sessions.h_end as s_h_text FROM teacher_schedule, groups, sessions ";
%   query += " WHERE day=" +week_day_global + " AND g_id=t_id_group AND t_id_session=s_id  ORDER BY t_id_session;";
%   console.log("querySchedulePerDayDB:" + query);
%   tx.executeSql(query,[], dbSuccessFunc = function(tx,rs){
%     queryScheduleSuccess(tx, rs);},
%       dbErrorFunc = function(tx, e) {
%     if (tx.message) e = tx;
%     alert(" There has been an error QuerySchedulePerDayDB: " + e.message);
%     return false;
%       });   
% }
% 
%  j) Add attendance, misbehaviour for each student. Student attendance, misbehaviour, punctuality are set here.
% Tasks to be done
%  k) Add error handling. Error handling is managed throw dbErrorFunc.
%  l) Retrieve and insert data from and to database. 
%  m) List of attendance, misbehaviour incidents. A list of attendance will be carried on, reusing Siestta interfaces.
%  n) Add activities grades for each student. 
%  o) List students marks and final mark. On a window, group, student name and surname will be shown, and a table with his/her marks.
%  p) Activities management window. Add, remove and update activities. It includes name of activity and percent weight.
%  q) Management of student notes. Student's notes could be included as an option.
%  r) List of student notes. In a table-like window, notes will be displayed.
% ).
% Make a group management window.
%  h) Students: These pages are not active.
% Make list of students window.
% Students management window (insert-update-delete students)
%  i) Timetable for actual date: list of groups for selected day.
% Below queryScheduleSuccess() and querySchedulePerDayDB() functions are written, these functions fills daily_schedule page as shown in Illustration 6: Application Skeleton.
% 
% /*
%  *  Main Window
%  */
% function queryScheduleSuccess(tx, results) {
%  var len = results.rows.length;
%    $('#groups_day_ul').empty();
%    var html;
%    var id=0;
%    var description="";
%    var start = "";
%    var t_id_session=-1;
%    for (var i=0;i<len;i++) {
%      id = results.rows.item(i).id;
%      t_id_session = results.rows.item(i).t_id_session;
%      start = results.rows.item(i).s_h_start;  
%      description = results.rows.item(i).description; 
%      html = "<li>";
%      html += "<div data-role='fieldcontain'>";
%      html += start;
%   html += "<a data-role='button' data-iconpos='notext' style='float: right;' href='index.html#list_students_attendance' ";
%    html += " onClick=\"listStudentsAttendance(" + results.rows.item(i).t_id_group + ","+t_id_session + ");\">" + description + "</a>";
%     html += "";
%     html += "</div>";
%     html += "</li>";
%    $('#groups_day_ul').append(html);
%    }
%    $('#groups_day_ul').listview('refresh');
%    }
% 
% /* Query groups per day - Main Window -
% */
% function querySchedulePerDayDB(tx){
%  var query = "SELECT teacher_schedule.id_session, teacher_schedule.day, teacher_schedule.id_group as t_id_group, teacher_schedule.id_session as t_id_session, ";
%   query += " groups.id as g_id, groups.data as description, sessions.id as s_id, sessions.h_start as s_h_start, sessions.h_end as s_h_text FROM teacher_schedule, groups, sessions ";
%   query += " WHERE day=" +week_day_global + " AND g_id=t_id_group AND t_id_session=s_id  ORDER BY t_id_session;";
%   console.log("querySchedulePerDayDB:" + query);
%   tx.executeSql(query,[], dbSuccessFunc = function(tx,rs){
%     queryScheduleSuccess(tx, rs);},
%       dbErrorFunc = function(tx, e) {
%     if (tx.message) e = tx;
%     alert(" There has been an error QuerySchedulePerDayDB: " + e.message);
%     return false;
%       });   
% }
% 
%  j) Add attendance, misbehaviour for each student. Student attendance, misbehaviour, punctuality are set here.
% Tasks to be done
%  k) Add error handling. Error handling is managed throw dbErrorFunc.
%  l) Retrieve and insert data from and to database. 
%  m) List of attendance, misbehaviour incidents. A list of attendance will be carried on, reusing Siestta interfaces.
%  n) Add activities grades for each student. 
%  o) List students marks and final mark. On a window, group, student name and surname will be shown, and a table with his/her marks.
%  p) Activities management window. Add, remove and update activities. It includes name of activity and percent weight.
%  q) Management of student notes. Student's notes could be included as an option.
%  r) List of student notes. In a table-like window, notes will be displayed
% 
% 
% 
% \end{verbatim}
% 


%%%%%%%%%%%%%%%%%%%%%%%%%%%%%%%%%%%%%%%%%%
%%%%%%%%%%%%%%%%%%%%%%%%%%%%%%%%%%%%%%%










\section{Database structure}
\begin{bclogo}[couleur=green!30,arrondi=0.1, logo=\bcpanchant,  ombre=true ] 
{EduXes Database structure}   
\begin{verbatim}
-- Groups
CREATE TABLE IF NOT EXISTS GROUPS (id  integer primary key ,
    data text , other_data text);
--- Students
CREATE TABLE IF NOT EXISTS STUDENTS (
      id integer primary key, id_group integer not null,
       name text, surname text,
      repeated integer, n_date text , photo text,
      tutor TEXT, address TEXT, phone text, e_phone text,
       nation text,
      FOREIGN KEY(id_group) REFERENCES GROUPS(id));
-- Sessions ( franja horaria)
CREATE TABLE IF NOT EXISTS SESSIONS (id  integer primary key,
        description text, h_start text, h_end text);
-- Teacher's schedule
CREATE TABLE IF NOT EXISTS TEACHER_SCHEDULE (id  integer primary key,
      id_session integer, day integer, id_group integer,
      FOREIGN KEY(id_group) REFERENCES GROUPS(id),
      FOREIGN KEY(id_session) REFERENCES SESSIONS(id));
-- Students Attendance
CREATE TABLE IF NOT EXISTS ATTENDANCE (id integer primary key ,
      id_group integer, id_student integer, id_session integer,
      a_type integer, a_date text,
      FOREIGN KEY (id_student) REFERENCES STUDENTS (id),
      FOREIGN KEY (id_group) REFERENCES GROUPS(id),
      FOREIGN KEY (id_session) REFERENCES SESSIONS(id) );
-- Activities
CREATE TABLE IF NOT EXISTS ACTIVITIES
    (id integer primary key, name text, date_init text, 
    date_end text, weight integer, final integer );
CREATE TABLE IF NOT EXISTS activities_student
    (id integer primary key ,  id_student integer,
     id_activity integer,
    mark integer, a_date text, notes text,
    FOREIGN KEY (id_student) REFERENCES students (id),
    FOREIGN KEY (id_activity) REFERENCES activities(id) );
CREATE TABLE IF NOT EXISTS activities_group
    (id integer primary key ,  id_group integer,
     id_activity integer,
    enabled integer, a_date text, notes text,
    FOREIGN KEY (id_group) REFERENCES groups (id),
    FOREIGN KEY (id_activity) REFERENCES activities(id) );

\end{verbatim}
\end{bclogo}

\section{Los algoritmos para el desarrollo de la soluci�n}
\dots

\section {qu� quieres resolver}
\dots
\section {c�mo lo vas a hacer}
\dots
\section {herramientas conceptuales necesarias}
 \dots
\section {Tools}
%% NOISE
NOISE vvvv


\begin{framed}
  ASD
\end{framed}
