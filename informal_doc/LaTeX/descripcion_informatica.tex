\chapter{Descripci�n Inform�tica (20-35\%)}

Para ello describir�s:

\section{La base de datos coleccionada (si tiene sentido).}
\dots
\section{Los algoritmos para el desarrollo de la soluci�n}
\dots

\section {qu� quieres resolver}
\dots
\section {c�mo lo vas a hacer}
\dots
\section {herramientas conceptuales necesarias}
 \dots
\section {Tools}
%\chapter{Anexo 2}
These are several tools used to build these documentation:
\begin {itemize}
\item Sqlfairy. Tranforms SQL language into a  image in \emph{png} format.
\item LaTeXila 2.7.0 to write this document. 1 
\item Gimp 2.8.2 to get screen-shots.
\item GNU/Debian Wheezy tools  October 2012 .
\end{itemize}



\begin{framed}
  ASD
\end{framed}

\begin{shaded}
  FFG
\end{shaded}

\begin{ovalbox} 
{
OVAL BOX
}
\end{ovalbox}

%%     la fleur : commande \bcfleur
%%     en chantier  : commande \bcpanchant (Jean-Michel SARLAT)
%%     la note : commande \bcnote (Thomas LABARRUSIAS)
%%     l etoile : commande \bcetoile
%%     l ourson : commande \bcours
%%     attention  : commande \bcattention
%%     le cour : commande \bccoeur
%%     ornement : commande \bcorne
%%     danger : commande \bcdanger (Fran�ois BOERKMANN)
%%     smiley heureux : commande \bcsmbh (Fran�ois BOERKMANN)
%%     smiley malheureux : commande \bcsmmh (Fran�ois BOERKMANN)
%%     Take care : commande \bctakecare (Patrick FRADIN)
%%     Lampe : commande \bclampe (Patrick FRADIN)
%%     Le livre : commande \bcbook (Patrick FRADIN)
%%     Le trefle : commande \bctrefle


\begin{shadowblock}{16cm}
OVAL BOX shadowblock
\end{shadowblock}

%% logo=\bcrayon,
\begin{bclogo} [couleur=blue!30,arrondi=0.1,  ombre=true ] 
{Contenido}
, contenido ...
\end{bclogo}
