\chapter{Descripci�n Inform�tica (20-35\%)}


Preparation of development:
 a) Build development environment: 
Java Development Kit (JDK) version 1.6 is downloaded from: 
http://www.oracle.com/technetwork/java/javase/downloads/index.html
As root user that file is unpacked into /usr/lib/jvm and configured to be the Java default:
\begin{verbatim}
 # update-java-alternatives -s JDK_1.6_NAME
\end{verbatim}  
	Eclipse Juno (4.6) is downloaded from its web-page:
http://www.eclipse.org Download->Linux 64 bits
Android Development Toolkit (ADT) is downloaded following instructions on this page:
http://developer.android.com/sdk/installing/installing-adt.html 
	A new line is included into repository software (Help ->  Install New Software -> Add):
http://dl-ssl.google.com/android/eclipse/
	Next step is to select all the related software listed.
	For Aptana Plugin line to add into Eclipse is: 
http://download.aptana.com/studio3/plugin/install
	Furthermore JQuery, JQueryMobile and Phonegap are needed, and were downloaded from their web sites:
JQuery 1.8.1 (no newer versions):
http://jquery.com/ 
JQuery will be  copied into assets/www/js folder.
JQueryMobile version 1.1.1 from
http://jquerymobile.com/ 
JQueryMobile is a zip file which will be uncompressed and copied into 
assets/www/js folder.
	PhoneGap - Cordova 1.8.1 is downloaded from this URL:
https://github.com/phonegap/phonegap/zipball/1.8.1 
	To create a PhoneGap application this very important instructions (Getting Started with Android)  should be followed step by step:
\cite{AndroidGettingStarted}

 b) Choose application name and folder's policy:
EduXes stands for "Educaci�n" and "Xesti�n", is a educational management software.
A folder is created (assets/www/js) which contains javascript (*.js) files except JQuery and JqueryMobile which is included into another folder (assets/www/js/jquery), do not forget style sheet files (*.css)
 c) Make a simple application: only a blank page.
Getting started with Android is followed step by step.
 d) Upload simple application into a git repository. 
A Github account is created and a new application is initialized.  This are the source code project page:\cite{EduXes}
Source code are upload to GitHub: 
  \begin{bclogo}[couleur=green!30,arrondi=0.1, logo=\bcpanchant,  ombre=true ] 
{Git init shell}   
\begin{verbatim}
$ git init
$ git add -A *
$ git remote add  EduXes git@github.com:joseantoniosa/EduXes.git 
$ git push origin master
\end{verbatim}
\end{bclogo}





\section{La base de datos coleccionada (si tiene sentido).}
\dots
\section{Los algoritmos para el desarrollo de la soluci�n}
\dots

\section {qu� quieres resolver}
\dots
\section {c�mo lo vas a hacer}
\dots
\section {herramientas conceptuales necesarias}
 \dots
\section {Tools}
%\chapter{Anexo 2}
These are several tools used to build these documentation:
\begin {itemize}
\item Sqlfairy. Tranforms SQL language into a  image in \emph{png} format.
\item LaTeXila 2.7.0 to write this document. 1 
\item Gimp 2.8.2 to get screen-shots.
\item GNU/Debian Wheezy tools  October 2012 .
\end{itemize}



\begin{framed}
  ASD
\end{framed}

\begin{shaded}
  FFG
\end{shaded}

\begin{ovalbox} 
{
OVAL BOX
}
\end{ovalbox}

%%     la fleur : commande \bcfleur
%%     en chantier  : commande \bcpanchant (Jean-Michel SARLAT)
%%     la note : commande \bcnote (Thomas LABARRUSIAS)
%%     l etoile : commande \bcetoile
%%     l ourson : commande \bcours
%%     attention  : commande \bcattention
%%     le cour : commande \bccoeur
%%     ornement : commande \bcorne
%%     danger : commande \bcdanger (Fran�ois BOERKMANN)
%%     smiley heureux : commande \bcsmbh (Fran�ois BOERKMANN)
%%     smiley malheureux : commande \bcsmmh (Fran�ois BOERKMANN)
%%     Take care : commande \bctakecare (Patrick FRADIN)
%%     Lampe : commande \bclampe (Patrick FRADIN)
%%     Le livre : commande \bcbook (Patrick FRADIN)
%%     Le trefle : commande \bctrefle


\begin{shadowblock}{16cm}
OVAL BOX shadowblock
\end{shadowblock}

%% logo=\bcrayon,
\begin{bclogo} [couleur=blue!30,arrondi=0.1,  ombre=true ] 
{Contenido}
, contenido ...
\end{bclogo}
