\chapter{State-of-art solutions}


	Only an open source application was found for study, Siestta  ,  nevertheless there are a lot of educational software (Sixa2, Unisoft3) but they are privative, Microsoft Windows freeware or both (SAS acad�mico4). 
	Siestta was evaluated. 
Technically it is an GPL'ed old style PHP-based web application with Ajax, an interactive editor, fckeditor and fpdf to generate reports.
From user point-of-view there are online documentation5. This application includes management of students, attendance, marks, tasks, incidents, general queries, letters to parents, interviews with parents, messages, appointments, exams and more.
	Several screen-shots were taken and will be reused in current application:


\section{Siestta}

This application (Siestta) are also available for PDAs, it could be a valid solution but it is server-side with outdated technologies. Data structure from Siestta is standard and fully functional, and it could be partially reused by EduXes.
Source code are also shown: calendario.php. It shows us a PHP application which uses sessions variables and is not Model-View-Controller oriented.

\fcolorbox{black}{gray!20}{
\parbox{\tw}{All parts of this prealgebra textbook are copyrighted � 2009 in the name Department of Mathematics, College of the Redwoods. They are not in the public domain. However, they are being made available free for use in educational institutions. This offer does not extend to any application that is made for profit. Users who have such applications in mind should contact David Arnold or Bruce Wagner at [hidden email] or [hidden email].
}} 


\begin{minipage}[c]{200}
 text text
 \end{minipage}




\begin {shaded}
\begin{verbatim}
<?php 
session_start(); 
require('config.php'); 
require('idioma/'.$idioma.''); 
include('funciones_calendario.php'); 
$docente = $_SESSION['usuario_sesion']; 
//recogemos variables 
$mes_actual = $_POST['mes']; 
$anyo_actual = $_POST['anyo']; 
if($mes_actual || $anyo_actual) { 
	include('funciones.php'); 
	conecta(); 
	} 
//si es la primera vez que entramos, cargamos la fecha actual 
if(!isset($mes_actual)) $mes_actual = date('m'); 
if(!isset($anyo_actual)) $anyo_actual = date('Y'); 
//presentamos ahora el calendario del mes actual o cargado 
//tabla con nombre mes y a�o y las flechas para navegar 
echo ' 
<br /> 
<table class="tablacentrada_i"> 
<tr> 
<td> 
<a href="#" onclick="navegaMes(\''.$mes_actual.'\',\''.$anyo_actual.
'\',\'menos\')" title="'.$id_anterior.'"><img src="imgs/anterior_peq.png" 
class="alin_bajo" alt="'.$id_anterior.'" /></a> 
'; 
$nombre_mes = numero_mes_a_nombre($mes_actual);
\end{verbatim}

\end{shaded}

Develop database structure: tables and relationships. 
Data base structure looks like Illustration 5: EduXes Database structure 


Bas�ndose en art�culos, libros, etc. que se os haya facilitado y
de otros que estim�is oportuno, se hablar� de:
\section{Descripci�n del problema}
\dots
\section{Descripci�n de los trabajos anteriores que se han dedicado a resolverlo}
\dots
