%%\chapter{Bibliography}
Lo relativo al anexo 1 \dots

Ejemplos de tratamiento de texto:

Probamos una cita \cite{NewCam97}\\
Citamos un libro\cite{Pesce}\\
Probamos a poner una nota al pi�\footnote{Mi primera nota al pi�}\\
Probando: \it{cursiva} \textbf{negrita} \underline{subrayada} \emph{enfatizar}\\


% pueden hacer falta varios anexos
%\chapter{Anexo 2}
Tools used:
Sqlfairy. Tranforms SQL language into a  png image.
LibreOffice 3.5.4.2 to write this document. 1 
Gimp 2.8.2 to get screen-shots.
GNU/ Debian Wheeze October 2012 

References:
Comparison among open source hosting facilities: http://en.wikipedia.org/wiki/Comparison_of_open_source_software_hosting_facilities
W3C Database Specifications: http://www.w3.org/TR/webdatabase/
PhoneGap Storage:  http://docs.phonegap.com/en/1.8.1/cordova_storage_storage.md.html
JqueryMobile: http://jquerymobile.com/demos/1.1.1/
Jquery: http://docs.jquery.com/
SergasApp: http://mrego.github.com/sergasapp/
EduXes: https://github.com/joseantoniosa/EduXes/
Siestta: http://siestta.sourceforge.net/doc/index.html
Xade Web: https://auth.edu.xunta.es/cas/login 

This document is hosted at:
https://github.com/joseantoniosa/EduXes/blob/master/docs/PhoneGap_Project_MSWL_Memory.odt
