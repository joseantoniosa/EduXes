\chapter{Results}

Application is evolving from list, edit students and groups, to its final goals. These objectives were fulfilled:
\section{Objectives completed}
\begin{itemize}
  \item Manage students and group of students. 
  \subitem List students and groups.
  \item Management of attendance and misbehaviour for each student. 
  \subitem List weekly attendance for each group.
  \item Management of assessment for each group and student. 
  \subitem List group assessment for each activity  
  \item Access to daily schedule to set assessments and attendance information.  
  \item Filter for only day of week days. 
  \item Test in real hardware: Android 4.0 and Android 2.3.3. Does not work with fluency in 4.0. It could be 
  caused because application is designed to Android 2.3.3
\end{itemize}


%%\section{Technical details}
%%Javascript-SQL conexion
%%Asynchronous methods


\section{Further objectives}

There are several objectives not fulfilled yet, those are, in priority order:
\begin{itemize}
\item Fix several bugs (blinking pages, assessments, etc.).
\item Test units. 
\item User Documentation.
\item Developer documentation. API documentation (JSDoc).
\item Add an image or photo to student.
\item Import data (students, groups, sessions, schedule, activities) from file or URL.
\item Timetable management. A window to manage groups timetable. When a group has class with this teacher.
\item Server synchronization with a custom application or XadeWeb \cite{Xade}.
\item Export data to a file.
\item Host application.
\item Upload to Google Play.
\item Xade web interface. It could be done through another an ad-hoc application.
\end{itemize}

And the most important objective is to build a community around this application. Firstly testers who help
to polish, add robustness  and  more functionality to EduXes.

These objectives were not fulfilled because lack of time, author's skills, too complicated to be achieved without a robust code,
and a community behind it.


\section {Problems faced}


Several problems were faced during this application development:

Eclipse environment: A stable, reliable and up-to-date IDE, with several plug-ins is needed. Download vanilla Eclipse Juno from its web-site was chosen because it is more stable, reliable, compatible with newer versions.  Aptana Javascript plugin was chosen because Aptana allows source code auto-completion in JQuery.

PhoneGap and Android incompatibilities. Android 2.3.3 requires JQuery-1.8.1 and does not work on higher versions. 


Error handlers. There were several problems with \textit{tx.executeSql(...)} function, it was confused with \textit{db.transaction(...)}: 
\begin{quote}
tx.executeSql(sql, [parameters],  successHandler, errorHandler)    
\end{quote}
and
\begin{quote}
db.transaction(queryFunction, errorHandler, successHandler)  
\end{quote}
have up to four and three parameters respectively, only first one is mandatory. First one was used because  success and error handlers for tx.executeSql allows an atomic error control.


Passing variables to functions: Only whether another solution is not known or feasible, global variables are used: named after \textit{global\_\*,} and there are several global variables in block capitals (for enumerators).

\section{Statistics}

With \textit{Ohloh} source code line counter, results are:
 \begin{bclogo}[couleur=green!30,arrondi=0.1, logo=\bcpanchant,  ombre=true ] 
{Ohloh counter}   
\begin{verbatim}
$ ohcount -i  assets/www/js/database.js assets/www/js/interface.js \
 assets/www/js/create_populate_db.js  assets/www/index.html \
 assets/www/remove.html 
\end{verbatim}
\end{bclogo}

\begin{shadowblock}{15cm}
 \begin{verbatim}
Examining 5 file(s)
                        Ohloh Line Count                              
Language   Code Comment Comment% Blank Total  File
-------- ------ ------- -------- ----- -----  ------------------
javascript 1306     148    10.2%   201  1655  database.js
javascript  388      84    17.8%    57   529  interface.js
javascript  402      57    12.4%    46   505  create_populate_db.js
html        535      42     7.3%    97   674  index.html
javascript    1       0     0.0%     0     1  index.html
html         28       1     3.4%     8    37  remove.html
\end{verbatim}
\end{shadowblock}

With David A. Wheeler's 'SLOCCount' statistics tools \cite{SLocCount}, results are (with Javascript patch\cite{SLocCountJS}):

 \begin{bclogo}[couleur=green!30,arrondi=0.1, logo=\bcpanchant,  ombre=true ] 
{ David A. Wheeler's 'SLOCCount'}   
\begin{verbatim}
$ /usr/local/bin/sloccount .
\end{verbatim}
\end{bclogo}

\begin{shadowblock}{15cm}
\begin{verbatim}
SLOC	Directory	SLOC-by-Language (Sorted)
1694    source_code     javascript=1694

Totals grouped by language (dominant language first):
javascript:      1694 (100.00%)

Total Physical Source Lines of Code (SLOC)               = 1,694
Development Effort Estimate, Person-Years (Person-Months)= 0.35 (4.17)
 (Basic COCOMO model, Person-Months = 2.4 * (KSLOC**1.05))
Schedule Estimate, Years (Months)                        = 0.36 (4.30)
 (Basic COCOMO model, Months = 2.5 * (person-months**0.38))
Estimated Average Number of Developers (Effort/Schedule) = 0.97
Total Estimated Cost to Develop                          = $ 46,989
 (average salary = $56,286/year, overhead = 2.40).
 
Generated using David A. Wheeler's 'SLOCCount'
\end{verbatim}
\end{shadowblock}
